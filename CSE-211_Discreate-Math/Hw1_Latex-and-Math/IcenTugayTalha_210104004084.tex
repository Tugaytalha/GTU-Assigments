\documentclass[a4 paper]{article}
\usepackage[inner=2.0cm,outer=2.0cm,top=2.5cm,bottom=2.5cm]{geometry}
\usepackage{setspace}
\usepackage[ruled]{algorithm2e}
\usepackage[rgb]{xcolor}
\usepackage{verbatim}
\usepackage{subcaption}
\usepackage{amsgen,amsmath,amstext,amsbsy,amsopn,tikz,amssymb,tkz-linknodes}
\usepackage{fancyhdr}
\usepackage[colorlinks=true, urlcolor=blue,  linkcolor=blue, citecolor=blue]{hyperref}
\usepackage[colorinlistoftodos]{todonotes}
\usepackage{rotating}
\usepackage{booktabs}
\usetikzlibrary{positioning,shapes,fit,arrows}
\newcommand{\ra}[1]{\renewcommand{\arraystretch}{#1}}

\newtheorem{thm}{Theorem}[section]
\newtheorem{prop}[thm]{Proposition}
\newtheorem{lem}[thm]{Lemma}
\newtheorem{cor}[thm]{Corollary}
\newtheorem{defn}[thm]{Definition}
\newtheorem{rem}[thm]{Remark}
\numberwithin{equation}{section}

\newcommand{\homework}[6]{
   \pagestyle{myheadings}
   \thispagestyle{plain}
   \newpage
   \setcounter{page}{1}
   \noindent
   \begin{center}
   \framebox{
      \vbox{\vspace{2mm}
    \hbox to 6.28in { {\bf CSE 211:~Discrete Mathematics \hfill {\small (#2)}} }
       \vspace{6mm}
       \hbox to 6.28in { {\Large \hfill #1  \hfill} }
       \vspace{6mm}
       \hbox to 6.28in { {\it Instructor: {\rm #3} \hfill Name: {\rm #5} \hfill Student Id: {\rm #6}} \hfill}
       \hbox to 6.28in { {\it Assistant: #4  \hfill #6}}
      \vspace{2mm}}
   }
   \end{center}
   \markboth{#5 -- #1}{#5 -- #1}
   \vspace*{4mm}
}

\newcommand{\problem}[2]{~\\\fbox{\textbf{Problem #1}}\hfill (#2 points)\newline\newline}
\newcommand{\subproblem}[1]{~\newline\textbf{(#1)}}
\newcommand{\D}{\mathcal{D}}
\newcommand{\Hy}{\mathcal{H}}
\newcommand{\VS}{\textrm{VS}}
\newcommand{\solution}{~\newline\textbf{\textit{(Solution)}} }

\newcommand{\bbF}{\mathbb{F}}
\newcommand{\bbX}{\mathbb{X}}
\newcommand{\bI}{\mathbf{I}}
\newcommand{\bX}{\mathbf{X}}
\newcommand{\bY}{\mathbf{Y}}
\newcommand{\bepsilon}{\boldsymbol{\epsilon}}
\newcommand{\balpha}{\boldsymbol{\alpha}}
\newcommand{\bbeta}{\boldsymbol{\beta}}
\newcommand{\0}{\mathbf{0}}


\begin{document}
\homework{Homework \#1}{Due: 30/10/22}{Dr. Zafeirakis Zafeirakopoulos}{Başak Karakaş}{}{}
\textbf{Course Policy}: Read all the instructions below carefully before you start working on the assignment, and before you make a submission.
\begin{itemize}
\item It is not a group homework. Do not share your answers to anyone in any circumstance. Any cheating means at least -100 for both sides. 
\item Do not take any information from Internet.
\item No late homework will be accepted. 
\item For any questions about the homework, send an email to bkarakas2018@gtu.edu.tr
\item Use LaTeX. You can work on the tex file shared with you in the assignment document.
\item Submit both the tex and pdf files into Homework1. Name of the files should be "\emph{SurnameName$\_$Id.tex}" and "\emph{SurnameName$\_$Id.pdf}".
\end{itemize}

\problem{1: Sets}{3+3+3+3+3=15}
Which of the following sets are equal? Show your work step by step.\newline
\subproblem{a} $\{$t : t is a root of $x^2$ – 6x + 8 = 0$\}$
\newline
\subproblem{b} $\{$y : y is a real number in the closed interval [2, 3]$\}$
\newline
\subproblem{c} $\{$4, 2, 5, 4$\}$
\newline
\subproblem{d} $\{$4, 5, 7, 2$\}$ - $\{$5, 7$\}$
\newline
\subproblem{e} $\{$q: q is either the number of sides of a rectangle or the number of digits in any integer between 11 and 99$\}$\\
\solution
\newline
$(a)$ \[ (x-4)×(x-2) = 0\]
\[ x = 4\]  \[ x=2\] 
\[ t\in\{4, 2\}\] 

$(b)$ \[ y\in[2,3]\]

$(c)$ \[ \{ 4,2,5,4\}\]

$(d)$ \[ \{ 4,2\}\]

$(e)$ The number of sides of a rectangle = 4, The number of digits in any integer between 11
and 99 = 2
\[ q\in\{ 4,2\}\]

 \centering\textit{\fontsize{15}{15}\fontfamily{ptm}\selectfont(a), (d) and (e) is equal}

\newpage
\problem{2: Cardinality of Sets}{2+2+2+2=8}
What is the cardinality of each of these sets? Explain your answers.\\
\subproblem{a} $\{\emptyset\}$\\
\subproblem{b} $\{\emptyset,\{\emptyset\}\}$\\
\subproblem{c} $\{\emptyset,\{\emptyset,\{\emptyset\}\}\}$\\
\subproblem{d} $\{\emptyset,\{\emptyset,\{\emptyset,\{\emptyset\}\}\}\}$\\
\solution
\newline
\newline
\newline
$(a)$ {Set that have 1 element that is a empty set ($\emptyset$)} \\
$(b)$ Have 2 different elements which are $\emptyset$ and $\{\emptyset\}$ \\
$(c)$ Have 2 different elements which are $\emptyset$ and $\{\emptyset,\{\emptyset\}\}$ \\
$(d)$ Have 2 different elements which are $\emptyset$ and $\{\emptyset,\{\emptyset,\{\emptyset\}\}\}$ \\~\\
The sets $(b)$, $(c)$ and $(d)$ have the element as a set. That is only one element, we can't count it separately
\newline
\newline
\newline




\problem{3: Cartesian Product of Sets}{15}
Explain why (A $\times$ B) $\times$ (C $\times$ D) and A $\times$ (B $\times$ C) $\times$ D are not the same.\\
\solution
\newline
\newline
{Assume that A = \{1\}, B = \{3, 4\}, C = \{5\}, D = \{7, 8\}} \\~\\
\raggedright{(a)} \[(A\times B) = \{(1, 3), (1, 4)\}\]
\[(C\times D) = \{(5, 7), (5, 8)\}\]
\[(A\times B)\times(C\times D) = ((1, 3), (5, 7)), ((1, 3), (5 ,8)), ((1, 4), (5, 7)), ((1 ,4), (5, 8))\]  \\~\\
\hspace{20}That means $(A\times B) = \{(a, b), a\in A, b\in B\}$ \hspace{20}
$(C\times D) = \{(c, d), c\in C, d\in D\}$
\[(A\times B)\times(C\times D) = \{((a, b),(c, d)), a\in A, b\in B, c\in C, d\in D\}\] 

\raggedright{(b)} \[(B\times C) = \{(3, 5), (4, 5)\}\]
\[A\times(B\times C)\times D = \{(1, (3, 5), 7), (1, (3, 5), 8), (1, (4, 5), 7), (1, (4, 5), 8)\}\]
\\~\\ \hspace{20}{That means $(B\times C) = \{(b, c), b\in B, c\in C\}$}
\[A\times (B\timesC)\times D) = \{(a, (b,c), d), a\in A, b\in B, c\in C, d\in D\}\] \\~\\
\hspace{20}$\{((a, b),(c, d))\}\neq\{(a, (b,c), d)\}$ therefore they aren't same 

\newpage
\problem{4: Cartesian Product of Sets in Algorithms }{25}
Let A, B and C be sets which have different cardinalities. Let (p, q, r) be each triple of A $\times$ B $\times$ C where p $\in$ A, q $\in$ B and r $\in$ C. Design an algorithm which finds all the triples that are satisfying the criteria: p $\leq$ q and q $\geq$ r. Write the pseudo code of the algorithm in your solution.\newline
\newline
For example: Let the set A, B and C be as A = $\{$ 3, 5, 7 $\}$, B = $\{$ 3, 6 $\}$ and C = $\{$ 4, 6, 9 $\}$. Then the output should be : $\{$ (3, 6, 4), (3, 6, 6), (5, 6, 4), (5, 6, 6) $\}$. \newline
\newline
(Note: Assume that you have sets of A, B, C as an input argument.)\newline
\solution

\begin{algorithm}
\SetAlgoLined
\KwIn{The sets of A, B, C}
\eIf{write a condition}{
    Statements
}{
 Statements
}
 When you want to write a for loop, you can use: \newline
\For{write a condition}{

}
 When you want to write a while loop, you can use: \newline
\While{write a condition}{
If you need to return, use \Return
}
 For any additional things you have to do while writing your pseudo code, Google "How to use algorithm2e in Latex?".
\caption{Pseudo Code of Your Algorithm}
\end{algorithm}

\begin{algorithm}
\SetAlgoLined
\KwIn{The sets of A, B, C}
$Rset \gets \{\}$ \\
\For{$p \in A$}{
    \For{$q \in B$}{
        \For{$r \in C$}{
            \If{$p \le q \And q \geq r$}{
                $Rset \gets Rset + \{(p, q, r)\}$
            }
        }
    }
}
\Return $Rset$

\caption{My Algorithm}
\end{algorithm}

\newpage
\problem{5: Functions}{16}
If f and f $\circ$ g are one-to-one, does it follow that g is one-to-one? Justify your answer.\\
\solution
\newline
\newline
Assume a function is one-to-one, if $f(x)=f(y) x=y$ and if $x\neq y, f(x)\neq f(y)$ \\~\\
\centering{if $g:A\xrightarrow[]{}B$ and $f:C\xrightarrow[]{}D$, $f\circ g:A\xrightarrow[]{}D$}
\\$f(z)$ is one-to-one and $f(g(x))$ is one-to-one, if $f(z)=f(g(x)), z = g(x)$ and if z = g(x) and $z\in C$,\hspace{3} $g(x)\in C$ \\~\\
{Let's assume $g(x)$ isn't one to one function when $g(x) = g(y)$ and $x\neq y$ \\ will be $f(g(x)) = f(g(y))$ but as we can see 2 elements of the domain correspond to a single element of the image set, that mean is if $g(x)$ isn't one-to-one $f\circ g(x)$ cannot be one-to-one therefore $g(x)$ is one-to-one}
\newline
\newline
\newline
\newline
\newline
\newline
\newline

\problem{6: Functions}{7+7+7=21}
Determine whether the function $f:$ $\mathbb{Z}\times\mathbb{Z}\to\mathbb{Z}$ is onto if\\
\subproblem{a} $f(m,n)=2m-n$\\
\subproblem{b} $f(m,n)=m^2-n^2$\\
\subproblem{c} $f(m,n)=\mid m\mid - \mid n\mid$\\
\solution
\newline

\raggedright{(a)}
Let's say $m = 0,$ \[ f(0,n) = 2*0 - n, \hspace{10} n\in Z\]
\begin{center}
\begin{tabular}{ | m{0.1cm} | m{0.1cm}| m{0.1cm} | } 
 \hline
 n & f(0,n) \\ [0.5ex] 
 \hline
 \hline
 -2 & 2  \\
 \hline
 -1 & 1\\
 \hline
 0 & 0 \\
 \hline
 1 & -1  \\
 \hline
 2 & -2  \\  [0.2ex] 
 \hline
\end{tabular}
\end{center}
$n\in Z$ for all $Z$ value $n$ have a value that's why this function is onto \\~\\

$(b)$ \\\hspace{20}There are no perfect square expressions with two spaces between them that's why $f(n,m) = m$ is not onto \\~\\

$(c)$ let's examine the problem in 2 parts  \\\hspace{10} 1. $f(m,0)$  \\\hspace{10} 2. $f(0,n)$ \\ 
$(1).$ \\\hspace{10} $f(m,0) = |m|,$ $m\in Z;$ $|m|$ is take all non-negative$(\{0\} + Z^+)$ values  
$(2).$ \\\hspace{10} $f(0,n) = -|n|$ $n\in Z;$ $|n|$ is always in $N$ therefore $-|n|$ take all non-positive$(\{0\} + Z^-)$ values  
\[ Z^- + \{0\} + Z^+ = Z\] as we can see  $f(m,n)$ takes all integer values and it is onto

\newpage
\problem{7: Functions}{Bonus 20}
Suppose that $f$ is a function from $A$ to $B$, where $A$ and $B$ are finite sets with $\mid A\mid=\mid B\mid$. Show that $f$ is one-to-one if and only if it is onto.\\
\solution
\newline
\newline
Onto is that all elements in the range match at least one element in the domain
and one-to-one is all elements in the range match at most one element in the domain \\ 
In the case where the cardinality of the sets is equal, since the number of elements in it are equal, for the function to be an non-matched element in the range (not onto), at least 2 elements in the domain must match the same 1 element in the image set, so it cannot be one-to-one
\begin{tikzpicture}[line width=1pt,>=latex]
\sffamily
\node (a1) {•1};
\node[below=of a1] (a2) {•2};
\node[below=of a2] (a3) {•3};
\node[below=of a3] (a4) {•4};

\node[right=4cm of a1] (aux1) {};
\node[below= 0.5cm of aux1] (b1) {•5};
\node[below=of b1] (b2) {•6};
\node[below=of b2] (b3) {•7};
\node[below=of b3] (b4) {•8};
\node[right=4cm of a4] (aux2) {};

\node[shape=ellipse,draw=myblue,minimum size=3cm,fit={(a1) (a4)}] {};
\node[shape=ellipse,draw=myblue,minimum size=3cm,fit={(aux1) (aux2)}] {};

\node[below=1.5cm of a4,font=\color{myblue}\Large\bfseries] {Domain};
\node[below=1.5cm of aux2,font=\color{myblue}\Large\bfseries] {Range};

\draw[->,myblue] (a1) -- (b1.170);
\draw[->,myblue] (a2) -- (b3.190);
\draw[->,myblue] (a3) -- (b2.175);
\draw[->,myblue] (a4.20) -- (b2.190);
\end{tikzpicture}

\end{document} 


